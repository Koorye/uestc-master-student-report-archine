\section{实验原理}

本章节将介绍实验中所用到的一些基本原理,包括Tiny Image表征、K-means聚类算法、KNN分类器、Bag of Words模型、倒排索引等。

\subsection{Tiny Image表征}

Tiny Image是一种简单的图像表征方法,它将图像缩放至固定大小,然后将图像展开为一个向量。Tiny Image表征的优点是简单易实现,但缺点是信息量较少,不适用于复杂的图像分类任务。

具体来说,Tiny Image表征算法可以表示为公式\ref{eq:tiny_image}:

\begin{equation}
    \label{eq:tiny_image}
    I_{tiny} = \text{Flatten}(\text{Resize}(I, (H_{out}, W_{out})))
\end{equation}

其中,$I$为原始图像,$I_{tiny}$为Tiny Image表征,$\text{Resize}(\cdot)$为图像缩放函数,$\text{Flatten}(\cdot)$为展平函数,将张量展平为一维向量,$H_{out}$和$W_{out}$为输出图像的高度和宽度。

一般来说,Tiny Image表征的大小为$16 \times 16$或$32 \times 32$。缩放的策略可以是不保持长宽比直接拉伸,也可以是保持长宽比,之后中心裁剪。最后,还需要对向量进行归一化处理,以投影到单位球面的度量空间上。

\subsection{K-means聚类算法}

K-means聚类算法是一种常见的无监督学习算法,用于将数据集划分为K个簇。K-means算法的目标是最小化数据点与其所属簇中心的距离之和,即最小化目标函数\ref{eq:kmeans}:

\begin{equation}
    \label{eq:kmeans}
    \min_{\mu_1, \mu_2, \ldots, \mu_K} \sum_{i=1}^{N} \min_{j} ||x_i - \mu_j||^2
\end{equation}

其中,$N$为数据点的数量,$x_i$为第$i$个数据点,$\mu_j$为第$j$个簇的中心点。该函数的含义是,通过学习合理的簇中心,之后对于每个数据点,找到其所属的最近的簇中心,使得数据点与簇中心的距离之和最小。

K-means算法的步骤可表示为算法\ref{alg:kmeans}:

% 用数学公式表示
\begin{algorithm}[H]
    \caption{K-means算法}
    \label{alg:kmeans}
    \begin{algorithmic}[1] %每行显示行号
        \Require 数据集$X$,簇的数量$K$
        \Ensure 簇中心$\mu_1, \mu_2, \ldots, \mu_K$
        \State 随机初始化簇中心$\mu_1, \mu_2, \ldots, \mu_K$
        \While{簇中心未收敛}
            \State 计算每个数据点$x_i$与簇中心$\mu_j$的距离$d_{ij}=||x_i - \mu_j||^2$
            \State 将数据点$x_i$划分到最近的簇中心$\mu_j$,$z_i = \arg\min_j d_{ij}$
            \State 更新簇中心$\mu_j = \frac{1}{|z_i|} \sum_{i=1}^{N} z_i x_i$
        \EndWhile
    \end{algorithmic}
\end{algorithm}

通过上述算法,可以得到数据集$X$的簇中心$\mu_1, \mu_2, \ldots, \mu_K$,从而实现数据集的聚类。虽然K-means算法简单易实现,但是需要事先指定簇的数量$K$,且对初始簇中心的选择较为敏感。

\subsection{KNN分类器}

KNN分类器是一种常见的监督学习算法,用于对数据点进行分类。KNN算法的基本思想是,对于一个新的数据点,找到与其最近的K个数据点,然后根据这K个数据点的类别,通过投票的方式决定新数据点的类别。具体来说,KNN算法的步骤可表示为算法\ref{alg:knn}:

\begin{algorithm}[H]
    \caption{KNN分类器}
    \label{alg:knn}
    \begin{algorithmic}[1] %每行显示行号
        \Require 训练数据集$X$,训练数据集的标签$Y$,新数据点$x$,K值
        \Ensure 新数据点$x$的类别
        \State 计算新数据点$x$与训练数据集中每个数据点$x_i$的距离$d_i=||x - x_i||$
        \State 找到与新数据点$x$最近的K个数据点$x_{i_1}, x_{i_2}, \ldots, x_{i_K}$
        \State 计算K个数据点的类别$y_{i_1}, y_{i_2}, \ldots, y_{i_K}$
        \State 通过投票的方式决定新数据点$x$的类别
    \end{algorithmic}
\end{algorithm}

其中投票的方式可以是简单多数投票,也可以是加权投票。KNN算法的优点是简单易实现,且对于多分类问题效果较好,但缺点是计算量较大,且需要事先指定K值。KNN分类器是一种惰性学习算法,即在训练阶段不需要学习,只需要存储训练数据集,然后在预测阶段对新数据点进行分类。然而,KNN算法依赖于距离度量,因此对于高维数据集效果较差。

\subsection{Bag of Visual Words模型}

在介绍Bag of Visual Words模型之前,先简单了解一下词袋模型(Bag of Words)。词袋模型是一种常见的文本表征方法,它通过统计文本中每个词的出现次数,将文本表示为一个词频向量。词袋模型的优点是简单易实现,但缺点是忽略了词语的顺序,且无法处理语义信息。受到词袋模型的启发,Bag of Visual Words模型将图像表示为一个视觉词频向量,Bag of Visual Words模型可以表示为算法\ref{alg:bovw}:

\begin{algorithm}[H]
    \caption{Bag of Visual Words模型}
    \label{alg:bovw}
    \begin{algorithmic}[1] %每行显示行号
        \Require 图像数据集$X=\{x_1,x_2,\dots,x_N\}$,KNN分类器$\text{KNN}(\cdot)$
        \Ensure $X$的类别$Y=\{y_1,y_2,\dots,y_N\}$
        \State 通过SIFT等算法提取图像数据集$X$的所有局部特征$F$,$f_i=f(x_i)$, 其中$f_i=\{f_{i1},f_{i2},\dots,f_{iM}\}$,$F=f_1\cup f_2\cup \dots \cup f_N$
        \State 对特征集合$F$进行聚类,得到视觉词典$V=\{v_1,\dots,v_K\}, K\ll N$
        \State $Y=\emptyset$
        \For {$x_i\in X$}
            \State 提取特征向量$f'_i=\{f'_{i1},f'_{i2},\dots,f'_{iM}\}$
            \State 量化局部特征到视觉词典中的最近词,得到集合$V_i'= \{v_{i1},v_{i2},\dots,v_{iM}\}$,其中$v_{ij}\in V$
            \State 统计词频向量$H=\{h_1,h_2,\dots,h_N\}$,其中$h_i=\text{Count}(V_i', v_{i})$
            \State 使用KNN分类器对词频向量$H_i$进行分类,$y_i=\text{KNN}(H_i)$
            \State $Y=Y \cup y_i$
        \EndFor
    \end{algorithmic}
\end{algorithm}

其中,$F$是所有图像局部特征集合的交集,$V$是$F$的聚类结果,$v_{ij}=arg\min_{v_k}||f_{ij}-v_k||$是词典中距离特征向量$f_{ij}$最近的词,$\text{Count}(A, a)$是统计函数,用于统计集合$A$中元素$a$的个数。Bag of Visual Words模型的优点是简单易实现,然而忽略了特征之间的空间关系。

\subsection{倒排索引}

倒排索引是一种常见的信息检索技术,用于快速查找包含某个关键词的文档。倒排索引的基本思想是,对于每个关键词,记录包含该关键词的文档列表。倒排索引的优点是快速查找,但缺点是需要较大的存储空间。倒排索引的构建过程可表示为算法\ref{alg:inverted_index}:

\begin{algorithm}[H]
    \caption{倒排索引}
    \label{alg:inverted_index}
    \begin{algorithmic}[1] %每行显示行号
        \Require 文档集合$D=\{d_1,d_2,\dots,d_N\}$
        \Ensure 倒排索引$I$
        \State $I=\emptyset$
        \For {$d_i\in D$}
            \State 提取文档$d_i$的关键词集合$K_i$
            \For {$k_j\in K_i$}
                \State $I[k_j]=I[k_j]\cup d_i$
            \EndFor
        \EndFor
    \end{algorithmic}
\end{algorithm}

通过上述算法,可以得到文档集合$D$的倒排索引$I$,可以快速通过关键词查找包含该关键词的文档。同样地,关键词集合可以用视觉词袋代替,从而实现图像快速检索。