\section{实验结论}

本次实验搭建了一个简单的光学神经网络模型,将图像转换到频域进行处理,然后再转换回空间域,该网络本质上是学习了一个快速傅立叶卷积操作。通过对MNIST数据集的训练,模型在训练结束时的准确率达到了94.9\%。光学神经网络模型的训练过程中,损失逐渐下降,准确率逐渐上升,说明模型在训练过程中逐渐收敛。光学神经网络模型的训练结果表明,光学神经网络模型可以实现对图像的处理,具有一定的应用前景。通过本次实验,我对光学神经网络模型的原理和实现有了更深入的了解,对深度学习和光学神经网络的研究有了更多的兴趣。

对于本次实验,我有以下建议:

\begin{itemize}
    \item 本次实验中,我使用了PyTorch框架搭建了光学神经网络模型,但是目前关于光学神经网络的资料较少,希望能够有更多的资料和教程,帮助更多的研究者了解和使用光学神经网络。
    \item 本次实验中,我使用了MNIST数据集进行训练,测试图像分类能力。希望能在更多任务上进行测试,如显著目标检测、图像重建等任务,验证光学神经网络模型的性能。
    \item 本次实验中,我搭建的光学神经网络模型较为简单,希望能够进一步优化模型结构,提高模型的性能,使其在更多任务上取得更好的效果。
\end{itemize}
