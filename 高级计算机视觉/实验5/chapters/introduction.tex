\section{实验内容和目的}

光学神经网络(ONN)是一种基于光学器件的神经网络,其基本原理是利用光学器件的并行性和高速性来加速神经网络的训练和推理过程。本实验将使用PyTorch实现一个简单的光学神经网络,并使用MNIST数据集进行训练和测试。实验目的是通过实现光学神经网络,加深对神经网络的理解,掌握PyTorch的基本使用方法,以及了解光学神经网络的基本原理。

本次实验的步骤有:

\begin{itemize}
    \item \texttt{student\_code.py}中四个函数的实现,代码以及相关解释。
    \item \texttt{proj5.ipynb}中训练流程以及MNIST分类准确度以及可视化分析。
    \item 思考光学神经网络通过梯度下降优化物理硬件参数的过程,理解并分析光学神经网络的推理过程,根据自己的理解,形成报告。
\end{itemize}

\section{实验环境}

本实验基于以下环境:

\begin{itemize}
    \item 操作系统:Windows 11
    \item 编程语言:Python 3.12.1
    \item 编程工具:Jupyter Notebook
    \item Python库:numpy 1.24.4、matplotlib 3.7.5、torch 2.2.2
\end{itemize}
