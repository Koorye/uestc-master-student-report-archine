\section{实验内容和目的}

特征点检测和匹配是计算机视觉中的一个重要问题,它是很多计算机视觉任务的基础,如目标识别、图像配准、三维重建等。

特征点检测的目的是在图像中找到一些具有显著特征的点,这些点在不同图像中具有一定的稳定性,常见的方法有Harris角点检测、SIFT、SURF等。而特征点匹配的目的是在两幅图像中找到对应的特征点,常见的方法有暴力匹配、K近邻匹配、RANSAC等。

针对特征点检测和匹配任务,本实验将完成以下内容:

\begin{enumerate}
    \item 学习和掌握Harris角点检测算法、SIFT算法。
    \item 实现特征点距离计算的函数。
    \item 实现特征点匹配的函数。
\end{enumerate}

通过上述内容,掌握特征点检测和匹配的基本原理和方法,为后续的计算机视觉任务打下基础。

\section{实验环境}

本实验基于以下环境:

\begin{itemize}
    \item 操作系统:Windows 11
    \item 编程语言:Python 3.12.1
    \item 编程工具:Jupyter Notebook
    \item Python库:numpy 1.24.4、matplotlib 3.7.5、torch 2.2.2
\end{itemize}
