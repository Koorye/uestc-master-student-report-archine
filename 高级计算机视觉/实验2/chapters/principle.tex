\section{实验原理}

本章节将介绍本次实验的原理,包括Harris角点检测算法、SIFT算法、特征点距离的计算和匹配。

\subsection{Harris角点检测算法}

Harris角度检测算法是一种经典的角点检测算法,它通过计算图像中每个像素点的角度响应函数,从而找到图像中的角点。角点是图像中的一种特殊的特征点,它在不同尺度和旋转下具有一定的稳定性。

Harris角点检测算法的核心思想是:对于一个角点,当图像在该点的水平和垂直方向上移动一个小的位移时,图像的灰度值会发生较大的变化。因此,可以通过计算图像在不同方向上的灰度变化来判断是否是角点。具体来说,可以表示为公式\ref{eq:harris}:

\begin{equation}
    E(u, v) = \sum_{x, y} w(x, y) [I(x + u, y + v) - I(x, y)]^2,
    \label{eq:harris}
\end{equation}

其中,$E(u, v)$表示在$(u, v)$处的角度响应函数,$w(x, y)$是一个窗口函数,$I(x, y)$是图像在$(x, y)$处的灰度值。当$E(u, v)$的值较大时,表示在$(u, v)$处可能是一个角点。

然而,直接计算$E(u, v)$的值是比较困难的,因为上述计算涉及到多层循环,计算量非常大。因此,Harris角点检测算法采用了一种近似的方法,对上述公式进行二阶泰勒展开,得到公式\ref{eq:harris-approx}:

\begin{equation}
    E(u, v) \approx \begin{bmatrix} u & v \end{bmatrix} M \begin{bmatrix} u \\ v \end{bmatrix},
    \label{eq:harris-approx}
\end{equation}

\begin{equation}
    M = \sum_{x, y} w(x, y) \begin{bmatrix} I_x^2 & I_x I_y \\ I_x I_y & I_y^2 \end{bmatrix},
\end{equation}

其中,$M$是一个$2 \times 2$的矩阵,表示在$(u, v)$处的梯度矩阵。通过计算$M$的特征值和特征向量,可以判断是否是角点。具体来说,当$M$的特征值较大时,表示在$(u, v)$处可能是一个角点。

根据主成分分析的原理,可以计算$M$的特征值和特征向量,然后根据特征值的大小来判断是否是角点。根据线性代数的知识,可以得到公式\ref{eq:harris-eigen}:

\begin{equation}
    R = \det(M) - \alpha \cdot \text{trace}(M)^2 = \lambda_1 \lambda_2 - \alpha (\lambda_1 + \lambda_2)^2,
\end{equation}

其中,$R$是Harris角点检测算法的响应值,$\lambda_1$和$\lambda_2$是$M$的特征值,$\alpha$是一个常数。当$R$的值较大时,表示在$(u, v)$处可能是一个角点。

上述公式可以理解为:如果$\lambda_1,\lambda_2$都很大,则表示该点上两个主方向上的梯度都很大,可能是一个角点;如果$\lambda_1,\lambda_2$的其中之一很大,则表示该点在一个方向上的梯度很大,可能是一个边缘;如果$\lambda_1,\lambda_2$都很小,则表示该点是一个平坦区域。

完整的Harris角点检测算法如下:

\begin{enumerate}
    \item 计算图像的梯度$I_x, I_y$。
    \item 计算梯度矩阵$M$。
    \item 计算响应值$R$,并根据阈值筛选角点。
    \item 非极大值抑制。
\end{enumerate}

通过上述算法,可以在图像中找到角点,从而实现特征点检测的目的。

\subsection{SIFT算法}

SIFT算法是一种经典的特征点检测和描述算法,它通过检测图像中的关键点,并计算关键点的特征向量,从而实现特征点的匹配。SIFT算法具有很好的尺度不变性和旋转不变性,因此在计算机视觉领域得到了广泛的应用。

SIFT算法的核心思想是:通过高斯金字塔和DoG金字塔来检测图像中的关键点,然后通过关键点周围的梯度信息来计算关键点的特征向量。具体来说,SIFT算法包括以下几个步骤:

\begin{enumerate}
    \item 构建高斯金字塔:通过不断降采样的方式构建高斯金字塔,得到不同尺度的图像。
    \item 构建DoG金字塔:通过高斯金字塔相邻两层图像相减得到DoG金字塔。
    \item 检测关键点:在DoG金字塔中检测局部极值点,得到关键点。
    \item 计算关键点的梯度:在关键点周围计算梯度信息。
    \item 计算关键点的特征向量:通过梯度信息计算关键点的特征向量。
\end{enumerate}

具体来说,高斯金字塔是通过不断降采样的方式构建的,每一层的图像是前一层图像高斯模糊后的降采样。DoG金字塔是通过高斯金字塔相邻两层图像相减得到的,用于检测局部极值点。

在得到DoG金字塔之后,可以通过非极大值抑制的方法来检测局部极值点,得到关键点。具体来说,对于每一个像素点,判断其与周围8个像素,以及相邻两层的9个像素的大小关系,如果是局部极值点,则认为是关键点。

在得到关键点之后,可以通过计算关键点周围的梯度信息来计算关键点的特征向量。具体来说,首先计算图像的梯度$I_x, I_y$,然后在关键点周围取若干区域,统计每个区域的梯度直方图,最后将梯度直方图拼接起来,得到关键点的特征向量。梯度直方图将梯度方向分为若干个区间,统计每个区间的梯度幅值,从而得到关键点的特征向量。

在得到关键点的特征向量之后,可以通过特征点距离的计算和匹配来实现特征点的匹配。具体来说,可以通过计算两个特征点之间的距离,然后通过一定的阈值来判断是否匹配。常见的方法有暴力匹配、K近邻匹配、RANSAC等。

通过上述算法,可以实现特征点的检测和匹配。

\subsection{特征点距离的计算和匹配}

特征点距离的计算和匹配是特征点检测和匹配的关键步骤,它决定了特征点匹配的准确性和鲁棒性。常见的方法有欧氏距离、余弦相似度、闵可夫斯基距离等。

欧氏距离是最常见的距离计算方法,它表示两个特征点之间的空间距离。具体来说,对于两个特征点$x, y$,可以通过公式\ref{eq:euclidean}来计算它们之间的欧氏距离:

\begin{equation}
    d(x, y) = \sqrt{\sum_{i=1}^{n} (x_i - y_i)^2},
    \label{eq:euclidean}
\end{equation}

其中,$x_i, y_i$表示两个特征点的第$i$个维度,$n$表示特征点的维度。

余弦相似度是一种用于计算向量之间相似度的方法,它表示两个特征点之间的相似度。具体来说,对于两个特征点$x, y$,可以通过公式\ref{eq:cosine}来计算它们之间的余弦相似度:

\begin{equation}
    d(x, y) = \frac{x \cdot y}{\|x\| \cdot \|y\|},
    \label{eq:cosine}
\end{equation}

其中,$x \cdot y$表示两个特征点的内积,$\|x\|, \|y\|$表示两个特征点的模长。

闵可夫斯基距离是一种通用的距离计算方法,它可以表示欧氏距离、曼哈顿距离、切比雪夫距离等。具体来说,对于两个特征点$x, y$,可以通过公式\ref{eq:minkowski}来计算它们之间的闵可夫斯基距离:

\begin{equation}
    d(x, y) = \left( \sum_{i=1}^{n} |x_i - y_i|^p \right)^{\frac{1}{p}},
    \label{eq:minkowski}
\end{equation}

其中,$p$是一个参数,表示闵可夫斯基距离的阶数。当$p=2$时,表示欧氏距离;当$p=1$时,表示曼哈顿距离;当$p=\infty$时,表示切比雪夫距离。

通过上述距离计算方法,可以计算两个特征点之间的距离,从而实现特征点的匹配。常见的方法有暴力匹配、K近邻匹配、RANSAC等。

暴力匹配是最简单的匹配方法,它通过计算所有特征点之间的距离,然后通过一定的阈值来判断是否匹配。具体来说,对于每一个特征点$x$,计算它与所有特征点$y$之间的距离,然后选择距离最小的特征点$y$作为匹配点。

K近邻匹配是一种更加有效的匹配方法,它通过计算每一个特征点的K个最近邻特征点,然后通过一定的阈值来判断是否匹配。具体来说,对于每一个特征点$x$,计算它与最近的K个特征点$y$之间的距离,然后根据比例筛选的原则,选择距离比例小于一定阈值的特征点$y$作为匹配点。

RANSAC是一种更加鲁棒的匹配方法,它通过随机采样的方式来估计模型参数,然后通过一定的阈值来判断是否匹配。具体来说,对于每一次随机采样得到的若干特征点构造一个模型,然后判断其余的特征点到模型的距离是否小于一定阈值,从而得到内点集合。再以得到的内点集合为基础,进行下一次随机采样。通过多轮迭代,可以得到最优的模型参数。

通过上述匹配方法,可以实现特征点的匹配,从而实现特征点检测和匹配的目的。