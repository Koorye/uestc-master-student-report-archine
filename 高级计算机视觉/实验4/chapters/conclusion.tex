\section{实验结论}

本实验通过深度学习对图像数据进行分类,实现了一个简单的图像分类器SimpleNet,并通过数据增强和Dropout操作对其进行了改进,缓解过拟合的问题。此外,本实验还实现了一个经典的卷积神经网络AlexNet,并对比了这两种模型的性能。从实验结果中可以看出,Dropout和数据增强可以缓解过拟合现象,此外设计合理的网络结构和预训练有利于提升模型的性能和泛化能力。通过本次实验,我对深度学习的图像分类任务有了更深入的理解,对深度学习的网络结构和训练技巧有了更多的实践经验。

对于本次实验,我有如下建议:

\begin{itemize}
    \item 实验中只使用了简单的Dropout操作,可以尝试更多的正则化方法,如L1正则化、L2正则化等,以缓解过拟合现象。
    \item 实验中只使用了简单的网络结构,可以尝试更复杂的网络结构,如ResNet、VGG等,以提升模型的性能和泛化能力。
\end{itemize}