\section{实验内容和目的}

深度学习是机器学习的一个分支,通过神经网络模型实现对数据的学习和预测,具体来说,深度学习通过设计网络结构和优化算法,实现对数据的特征提取和模型训练,从而实现对数据的预测和分类。本实验通过PyTorch框架实现一个简单的神经网络模型,通过训练数据集,实现对数据的分类任务。

具体来说,本实验包含以下内容:

\begin{itemize}
    \item 通过PyTorch实现数据集的加载和预处理;
    \item 使用PyTorch搭建一个简单的神经网络模型;
    \item 编写预测和损失计算函数;
    \item 使用数据增强和Dropout技术,提高模型的泛化能力;
    \item 测试AlexNet的性能。
\end{itemize}

\section{实验环境}

本实验基于以下环境:

\begin{itemize}
    \item 操作系统:Windows 11
    \item 编程语言:Python 3.12.1
    \item 编程工具:Jupyter Notebook
    \item Python库:numpy 1.24.4、matplotlib 3.7.5、torch 2.2.2
\end{itemize}
