\section{介绍}

% 介绍深度学习的发展背景
随着深度学习的发展,近年来大量模型被提出和应用,如ResNet\cite{he2016deep}、BERT\cite{devlin2018bert}等。这些模型在计算机视觉、自然语言处理等领域取得了巨大的成功。同时,大模型的出现也使得深度学习的应用范围更加广泛。如ChatGPT、GPT-4等模型在对话生成、文本生成等任务上取得了很好的效果。同时,研究者还利用这些大模型来辅助解决复杂任务,如自动驾驶\cite{fu2024drive}、医疗诊断\cite{lievin2023can}等,大模型的出现使得深度学习的应用范围更加广泛。

% 介绍大模型的计算和存储成本问题
然而,大模型的出现也带来了一些问题,如计算和存储成本问题。大模型的训练和推理需要大量的计算资源,例如,GPT-4的训练中使用了约25,000个 A100 GPU,耗费90到100天,训练一次的成本就达到6300万美元。另外,大模型的存储也是一个问题。大模型的存储和加载需要大量的内存,而这些内存也是非常昂贵的。此外,大模型也涉及到安全、隐私问题,如用户的数据可能会被泄露。因此,大模型训练和部署的成本和安全性问题亟待解决。

% 介绍移动边缘技术
移动边缘技术\cite{mao2017survey,abbas2017mobile}是一种新兴的技术,它将计算资源和存储资源放在网络边缘,使得用户可以更快地访问数据和服务。移动边缘技术可以提高用户体验,减少网络延迟,降低网络拥塞。同时,移动边缘技术还可以提高网络的安全性,保护用户的隐私。因此,移动边缘技术在智能手机、物联网、自动驾驶等领域有着广泛的应用。

% 介绍联邦学习
目前,研究者们已经开始研究如何将深度学习模型部署到移动边缘设备上。联邦学习\cite{zhang2021survey,mammen2021federated}是一种将深度学习模型部署到移动边缘设备上的方法,本质上是通过多个用户设备共同训练一个代表所有用户设备的全局模型,而训练的过程不需要用户数据的交换。这种方法可以在不泄露用户数据的情况下,训练深度学习模型,并减少数据传输的开销、计算和存储成本等问题,联邦学习的优势使得它在移动边缘设备上有着广泛的应用前景。

截至目前,联邦学习作为一种隐私保护的重要解决方案,已在全球范围内得到广泛关注和应用。在联邦学习领域,中国和美国是联邦学习论文发布量最多的两个国家,高被引论文的六成以上来自这两国。此外,各国研究者提出了各种联邦学习开源框架,其中OpenMined推出的Pysyft\cite{ziller2021pysyft}、微众银行的FATE\cite{kholod2020open}和谷歌的TFF框架热度居于全球前三位,这些框架为联邦学习的研究和应用提供了重要的支持。此外,各国政府和企业也在积极推动联邦学习的发展,如美国的联邦数据战略、中国的数据安全法等,这些政策和法规为联邦学习的发展提供了政策支持。

% 介绍本章内容
本文将介绍联邦学习技术的背景,一些联邦学习的方法,以及联邦学习在一些应用场景中的应用,最后讨论联邦学习的优缺点和未来发展方向。