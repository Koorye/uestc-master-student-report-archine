\section{背景}

这一部分将介绍联邦学习技术的一些背景知识,包括深度学习和移动边缘技术的发展历史。

\subsection{深度学习的发展历史}

% 介绍深度学习的发展历史,AlexNet、ResNet、BERT、GPT系列等
深度学习是机器学习的一个分支,它模仿人类大脑的神经网络,通过多层神经元进行信息处理。深度学习的发展可以追溯到上世纪50年代,但直到2012年,AlexNet的出现才使得深度学习开始受到广泛关注。AlexNet\cite{deng2009imagenet}是一种卷积神经网络,它在ImageNet\cite{deng2009imagenet}图像识别比赛中取得了巨大的成功。2016年,ResNet模型的出现进一步推动了深度学习的发展。ResNet是一种残差神经网络,它通过残差块的设计,解决了深度神经网络训练过程中的梯度消失和梯度爆炸问题。2018年,BERT模型的出现进一步推动了自然语言处理领域的发展。BERT是一种预训练的语言模型,它在多项自然语言处理任务上取得了很好的效果。2019年,GPT-2模型的出现进一步推动了文本生成领域的发展。GPT-2\cite{radford2019language}是一种基于Transformer的语言模型,它在文本生成任务上取得了很好的效果。这些模型的出现使得深度学习在计算机视觉、自然语言处理等领域取得了巨大的成功。

\subsection{移动边缘技术的发展历史}

% 介绍移动边缘技术的发展历史,MEC、Fog Computing等
移动边缘技术是一种新兴的技术,它将计算资源和存储资源放在网络边缘,使得用户可以更快地访问数据和服务。移动边缘技术最早可以追溯到MEC(Mobile Edge Computing)\cite{shi2016edge}技术,它是一种将计算资源和存储资源放在无线接入网的边缘,使得用户可以更快地访问数据和服务。MEC技术可以提高用户体验,减少网络延迟,降低网络拥塞。另外,Fog Computing\cite{bonomi2012fog}技术也是一种移动边缘技术,它是一种将计算资源和存储资源放在网络边缘的技术,使得用户可以更快地访问数据和服务。Fog Computing技术可以提高用户体验,减少网络延迟,降低网络拥塞。移动边缘技术在智能手机、物联网、自动驾驶等领域有着广泛的应用。