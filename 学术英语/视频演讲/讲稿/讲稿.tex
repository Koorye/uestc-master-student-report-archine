\documentclass{article}
\usepackage{geometry}

\geometry{a4paper,scale=0.85}

\title{Development and Application of GPT}
\date{December 11, 2023}
\author{Koorye}

\begin{document}
\maketitle
\thispagestyle{empty}

Hello everyone, my name is Koorye, my major is computer science and technology, and I am currently a beginner in deep learning research. Today, I want to introduce you to a creative product - GPT, which has brought great changes to our lives. Let's dive into the fascinating journey of GPT models, from their inception to the latest iterations. 

\textbf{GPT (Generative Pre-trained Transformer)} is a large-scale language model developed by OpenAI. It is capable of generating human-like text based on context and past conversations. GPT has a range of products including GPT-1, 2, 3 and 4. I'll introduce you one by one.

\textbf{GPT-1} is the earliest language model of the GPT family, which was developed in 2018. GPT-1 is relatively small in scale compared to subsequent versions, but it offers a way to pre-train large text and has the ability to generate continuous text based on context.

In 2019, OpenAI developed \textbf{GPT-2}. Compared to GPT-1, GPT-2 has larger parameters and more diverse datasets, which makes it more powerful for text generation.

\textbf{GPT-3} has a thousand times more parameters than GPT-2. Based on GPT-3, a number of applications have been developed, such as ChatGPT for conversation and CodeX for writing code. To date, there are more than 300 applications using GPT-3 in different fields and industries, from productivity and education to creativity and gaming.

In March 2023, \textbf{GPT-4} was born. GPT-4 introduces multi-modal capabilities, that is, GPT-4 can accept not only text input, but also image input. This gives it a higher level of complex reasoning. GPT-4 can cope with complex tasks that require both images and text, such as answering questions based on images. The figure shows GPT-4 compared to previous models, with its performance reaching new levels in tests such as mathematics, physics, foreign languages, and law.

GPT is a powerful language model that can be used for many different tasks and applications. Here are some uses of GPT:

\begin{enumerate}

\item\textbf{Natural language generation}: GPT can generate natural and fluent text, including articles, stories and poems.
\item\textbf{Dialog system}: GPT can hold conversations with users, answer questions, offer suggestions.
\item\textbf{Text summary and translation}: GPT can extract key information from long texts, generate summaries, or translate text from one language to another.
\item\textbf{Code generation}: GPT can automatically write code that conforms to specifications and logic according to the needs of users, supporting a variety of programming languages and frameworks.
\item\textbf{Image analysis}: GPT can accept images as input and generate titles, classifications, and analyses that show the content and features of the images.
\item\textbf{Content production}: GPT can generate various types of content according to user requirements, such as articles, abstracts, reviews, poems, lyrics, stories, games, etc., with rich creativity and imagination.

\end{enumerate}

Although GPT has powerful reasoning and generative capabilities, however, it also has many limitations. For example, GPT also produces false predictions that are difficult to correct. In addition, GPT lacks a sociocultural and ethical assessment, and thus may generate biased responses. For example, when asked to describe a picture of the solar system with Saturn hidden, the GPT automatically assumed that Saturn was in the picture, behind Jupiter. Therefore, we cannot abuse GPT and other AI technologies. We must regulate their use under appropriate supervision in order to make better contributions to humankind.

In summary, the GPT family—GPT-1, GPT-2, GPT-3 and GPT-4—represents a remarkable evolution in natural language understanding. These models have transformed how we interact with AI, making complex tasks seem almost magical. Thank you!
 
\end{document}