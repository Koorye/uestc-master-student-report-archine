\documentclass{article}
\usepackage[UTF8]{ctex}
\usepackage{amsmath}
\usepackage{fancyhdr}
\usepackage{geometry}
\usepackage{sectsty}

\geometry{left=2cm, right=2cm, top=2cm, bottom=2cm}
\sectionfont{\fontsize{12pt}{15pt}\selectfont}

\title{机器学习——第一次作业}
\author{Koorye}
\date{2024年3月2日}

\pagestyle{fancy}
\fancyhead[L]{机器学习——第一次作业}
\fancyhead[R]{Koorye}
\fancyfoot[R]{基于\LaTeX 排版制作}

\begin{document}
\maketitle
\thispagestyle{fancy}

\section{过学习有什么表现?原因是什么?有哪些避免过学习的方法?}

过学习表现为模型在训练集上的误差很低,但是在测试数据上误差很高。

过学习的原因是训练数据太少,或模型能力太强(参数过多、结构过于复杂)等。

避免过学习的方法有:
\begin{enumerate}
    \item 扩大训练集:更多训练数据可以使模型拥有更为平滑的决策面。
    \item 正则化:修改目标函数来惩罚模型复杂度,如L1/L2正则化等,目的是约束模型的学习能力。
    \item 通过验证集来选择模型:通过验证集确定模型超参数,还可以通过K折交叉验证、留一法等在数据不足的情况下更好地评估模型性能。
\end{enumerate}

\section{监督学习中,什么是推断,什么是决策,简述解决决策问题的三种方法及其复杂度。}

推断指利用训练数据学习后验概率$p(t|x)$或$p(x,t)$,使用已知类别的样本,找到输入$x$和输出结果$t$之间的关系。

决策指使用后验概率进行最优分类,将输入$x$映射到最优分类结果$t$。

解决决策问题的三种方法有:
\begin{enumerate}
    \item 生成式模型:通过贝叶斯定理求出后验概率$p(t|x)=\frac{p(x|t)p(t)}{p(x)}$,再进行决策。复杂度高。
    \item 判别式模型:直接得到条件概率$p(t|x)$进行决策。复杂度中等。
    \item 判别函数:直接得到判别函数$f(x)$,将输入$x$映射到最优分类结果$t$。复杂度低。
\end{enumerate}

\section{(贝叶斯定理)一种癌症,得了这个癌症的人被检测出为阳性的几率为80\%,未得这种癌症的人被检测出阴性的几率为90\%,而人群中得这种癌症的几率为1\%,一个人被检测出阳性,问这个人得癌症的几率为多少?}

设事件$A$为得癌症,事件$B$为检测出阳性。则有:

\begin{equation}
    P(A|B)=\frac{P(B|A)P(A)}{P(B)}=\frac{P(B|A)P(A)}{P(B|A)P(A)+P(B|\sim A)P(\sim A)}.
\end{equation}

由已知条件得

\begin{equation}
    P(A)=0.01, P(B|A)=0.8, P(\sim B|\sim A)=0.9.
\end{equation}

则有

\begin{equation}
    P(\sim A)=0.99, P(B|\sim A)=0.1.
\end{equation}

代入得

\begin{equation}
    P(A|B)=\frac{0.8\cdot 0.01}{0.8\cdot 0.01+0.1\cdot 0.99}\approx 0.0748.
\end{equation}

故约有7.48\%的概率得癌症。

\section{什么是极大似然参数估计和最大后验参数估计,简述它们的特点和联系,并说明什么是共轭先验。}

极大似然估计(MLE)指在给定观测数据$D$的情况下,调整参数$\theta$使得似然函数$p(D|\theta)$最大。

\begin{equation}
    \hat{\theta}=arg\max_{\theta}P(D|\theta).
\end{equation}

最大后验估计(MAP)指在给定观测数据$D$的情况下,调整参数$\theta$使得后验概率$p(\theta|D)$最大。

\begin{equation}
    \hat{\theta}=arg\max_{\theta}P(\theta|D)=arg\max_{\theta}\frac{P(D|\theta)P(\theta)}{P(D)}.
\end{equation}

极大似然估计的特点是只考虑样本数据,不考虑先验信息;最大后验估计的特点是考虑了先验信息。因此极大似然估计在数据量足够时才能比较准确,数据不足时容易产生过拟合;而最大后验估计可以通过先验信息来约束参数的取值,避免过拟合。

两者的联系是当先验分布为均匀分布时,最大后验估计退化为极大似然估计;当数据足够多时,两者的结果也会趋于一致。

共轭先验指在贝叶斯估计中,如果后验分布和先验分布属于同一分布族,则称先验分布是似然函数的共轭先验。例如,将二项分布的先验设为Beta分布,即

\begin{equation}
    P(\theta)=\text{Beta}(\alpha,\beta)=\frac{\Gamma(\alpha+\beta)}{\Gamma(\alpha)\Gamma(\beta)}\theta^{\alpha-1}(1-\theta)^{\beta-1},
\end{equation}

有似然

\begin{equation}
    P(x|n,\theta)=\binom{n}{x}\theta^x(1-\theta)^{n-x},
\end{equation}

则后验

\begin{equation}
    P(\theta|x,n,\alpha,\beta)\propto\text{Beta}(\alpha+n,\beta+N-n).
\end{equation}

\end{document}