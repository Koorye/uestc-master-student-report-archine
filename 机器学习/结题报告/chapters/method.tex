\section{方法}

本章节将介绍本文使用的一些方法,包括基于全局和局部特征的分析、基于降维可视化的分析、模型性能和泛化性的比较等。

\subsection{基于注意力范围分析}

Grad-CAM(Gradient-weighted Class Activation Mapping)是一种用于可视化深度学习模型的类激活图(CAM)的技术,它通过利用模型中的梯度信息来生成区分不同类别的图像区域,能够准确地定位模型在做出分类决策时所关注的是图像哪部分区域,这有助于解释模型的预测结果、理解模型的决策过程以及识别模型的弱点。

Grad-CAM的基本原理和步骤如下:

1. 前向传播
首先,将图像输入深度学习模型中进行前向传播,得到模型的输出,此项目中是图像分类任务。

2. 反向传播
接着,计算目标类别对于模型最后一层特征图的梯度。其中目标类别是可以进行选择的,可使模型针对想要观察的类别标签进行梯度的计算。

3. 特征图权重计算
将目标类别的梯度与模型最后一层特征图进行加权相加,得到每个特征图的权重。其中权重反映了每个特征图对于目标类别的重要性。

4. 生成类激活图
将特征图与对应的权重相乘并求和,得到最终的类激活图。这个类激活图表示了模型在做出分类决策时所关注的图像区域。

5. 可视化
将类激活图叠加到原始输入图像上,产生彩色的热力图。热力图上的颜色表示了模型在分类决策中所关注的区域。通常,更亮的颜色对应于模型更关注的区域。

Grad-CAM的优点在于它不需要对模型进行修改,只需在前向传播和反向传播中捕获梯度信息即可。因此,Grad-CAM适用于各种深度学习模型和任务,并且能够提供直观且解释性强的可视化结果。Grad-CAM已被广泛应用于图像分类、目标检测、图像分割等领域,以帮助深度学习模型的理解和调试。
\subsection{基于全局和局部特征的分析}
分析全局和局部特征通常需要对神经网络的隐藏层输出特征进行定量的计算。而分析神经网络的(隐藏)层表示往往具有挑战性,因为它们的特征分布在大量神经元上。这种分布式方面也使得跨神经网络的表示难以进行有意义的比较。中心内核对齐(CKA)解决了这些挑战,实现了网络内部和网络之间表征的定量比较。
通常CKA算法将$x\in \mathbb{R}^{m\times p_{1}}$以及$y\in \mathbb{R}^{m\times p_{2}}$作为输入,其中$m,p_{1},p_{2}$分别表示样本的个数,以及两层网络神经元的个数。定义$K=XX^{T}$以及$L=YY^{T}$作为两个网络层的$Gram$矩阵,$CKA$计算如下:
$$CKA(K,L)=\frac{HSIC(K,L)}{\sqrt{HSIC(K,K)HSIC(L,L)} } $$其中$HSIC$是希尔伯特-施密特独立准则。CKA对表示的正交变换(包括神经元的置换)是不变性的,其归一化项保证了对各向同性尺度的不变性。这些特性使得神经网络隐藏表征的比较和分析变得有意义。

同时为了去研究两种模型$ViT$和$ConvNeXt$之间关于高低频信息的处理能力,我们通过傅里叶变换对图像特征图进行分析。图像频域分析是一种将图像从空间域(空间坐标)转换到频域(频率坐标)的方法。在频域中,图像表示为频谱的形式,反映了图像中不同频率的分量。傅立叶变换是一种常用的频域分析方法,可以将图像从空间域转换到频率域。傅立叶变换将图像表示为频率分量的幅度和相位信息,使得图像在频域中的特征更加明显。傅立叶变换在图像处理中被广泛应用,如图像压缩、去卷积、特征提取等。对于一个大小为$M\times N$的图像傅里叶变换公式如下:
$$F(u, v)=\sum_{x=0}^{M-1} \sum_{y=0}^{N-1} f(x, y) e^{-i 2 \pi\left(\frac{u x}{M}+\frac{v y}{N}\right)}$$

其中$f(x,y)$表示图像在空间域中的像素强度,$F(u,v)$表示图像在频率域中的频率分量,$u$和$v$是频率索引。傅里叶变换将图像从空间域转换到频率域,使得图像在频域中的特征更加明显。通过对图像的频域分析,可以更好地理解图像的特征和结构,为图像处理和分析提供更多的信息。

\subsection{基于降维可视化的分析}

降维可视化技术是理解和比较高维数据特征的一种有效工具,特别是在模型特征表示上的理解。通过将高维特征空间映射到二维或三维空间,可以观察和比较卷积神经网络(CNN)和视觉变压器(ViT)模型提取的特征之间的差异。本文使用了一种流行的非线性降维技术:t-Distributed Stochastic Neighbor Embedding(t-SNE)。

t-SNE是一种高级降维技术,与PCA相比,它更擅长保留数据的局部结构,并在低维空间中展示数据点之间的相对关系。在使用t-SNE进行降维之前,通常会进行数据的标准化处理,以确保不同的特征尺度不会影响降维的结果。t-SNE的核心思想是在低维空间中寻找一个数据点的分布,这个分布可以最好地反映高维空间中数据点之间的相似性。

\subsection{模型性能和泛化性的比较}

在传统图像分类任务上,在ImageNet数据集上测试准确率是评估模型性能的主要方法,自从2012年AlexNet首次在ImageNet上取得突破性进展以来,ImageNet准确率一直是评估模型性能的重要指标。然而,随着大模型的出现和数据集的增大,ImageNet准确率已经不能完全反映模型的性能。首先,在大模型选择在更大、更复杂的数据集上进行预训练之后,ImageNet已经是一个相对简单的数据集,其类别数量和复杂程度已经不能满足真实世界的需求。其次,ImageNet准确率不能反映模型的泛化能力,即模型在未见过的数据上的表现。

为了探究模型性能受哪些因素影响,本文选择了ImageNet-X和PUG-ImageNet 2种数据集,这两个数据集提供了丰富的属性注释,可以通过控制变量法来定量研究模型对于哪些属性更加敏感。ImageNet-X数据集是ImageNet数据集的扩展,其中包含了一些常见的数据扰动,如模糊、噪声、对比度变化等。PUG-ImageNet数据集是一个人为构造的数据集,其中包含了一些特定属性,如颜色、形状、纹理等。通过这两个数据集,本文可以研究模型对于不同属性的敏感性,从而揭示模型的内在特性。

为了探究模型的泛化性能,本文选择了跨域、跨数据集和图像变换等任务。跨域任务是指模型在一个数据集上训练,然后在另一个数据集上测试,这可以评估模型对于不同数据分布的适应能力。跨数据集任务是指模型在一个数据集上训练,然后在另一个数据集上测试,这可以评估模型对于不同数据集的泛化能力。图像变换任务是指模型在一个数据集上训练,然后在另一个数据集上测试,这可以评估模型应对图像变换或破坏时的适应能力。通过这些任务,本文可以研究模型的泛化能力,从而揭示模型的通用性。
