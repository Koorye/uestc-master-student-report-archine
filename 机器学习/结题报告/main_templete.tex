%% 该模板修改自《计算机学报》latex 模板
%% 主要是将双栏改成单栏,去掉了部分计算机学报标识;
%% 源文件自:https://www.overleaf.com/latex/templates/latextemplet-cjc-xelatex/ybmmymncrrmw

\documentclass[10.5pt,compsoc,UTF8]{CjC}
\usepackage{CTEX}
\usepackage{graphicx}
\usepackage{float}
\usepackage{footmisc}
\usepackage{subfigure}
\usepackage{url}
\usepackage{multirow}
\usepackage{multicol}
\usepackage[noadjust]{cite}
\usepackage{amsmath,amsthm}
\usepackage{amssymb,amsfonts}
\usepackage{booktabs}
\usepackage{color}
\usepackage{ccaption}
\usepackage{booktabs}
\usepackage{float}
\usepackage{fancyhdr}
\usepackage{caption}
\usepackage{xcolor,stfloats}
\usepackage{comment}
\setcounter{page}{1}
\graphicspath{{figures/}}
\usepackage{cuted}%flushend,
\usepackage{captionhack}
\usepackage{epstopdf}
\usepackage{gbt7714}
\usepackage{newclude}
\usepackage{subfloat}

%===============================%

\headevenname{\mbox{\quad} \hfill  \mbox{\zihao{-5}{ \hfill 《机器学习》  } \hspace {50mm} \mbox{2024 年 4 月}}}%
\headoddname{ \hfill 卷积神经网络和ViT的比较}%

%footnote use of *
\renewcommand{\thefootnote}{\fnsymbol{footnote}}
\setcounter{footnote}{0}
\renewcommand\footnotelayout{\zihao{5-}}

\newtheoremstyle{mystyle}{0pt}{0pt}{\normalfont}{1em}{\bf}{}{1em}{}
\theoremstyle{mystyle}
\renewcommand\figurename{figure~}
\renewcommand{\thesubfigure}{(\alph{subfigure})}
\newcommand{\upcite}[1]{\textsuperscript{\cite{#1}}}
\renewcommand{\labelenumi}{(\arabic{enumi})}
\newcommand{\tabincell}[2]{\begin{tabular}{@{}#1@{}}#2\end{tabular}}
\newcommand{\abc}{\color{white}\vrule width 2pt}
\renewcommand{\bibsection}{}
\makeatletter
\renewcommand{\@biblabel}[1]{[#1]\hfill}
\makeatother
\setlength\parindent{2em}
%\renewcommand{\hth}{\begin{CJK*}{UTF8}{gbsn}}
%\renewcommand{\htss}{\begin{CJK*}{UTF8}{gbsn}}

\begin{document}

\hyphenpenalty=50000
\makeatletter
\newcommand\mysmall{\@setfontsize\mysmall{7}{9.5}}
\newenvironment{tablehere}
  {\def\@captype{table}}

\let\temp\footnote
\renewcommand \footnote[1]{\temp{\zihao{-5}#1}}


\thispagestyle{plain}%
\thispagestyle{empty}%
\pagestyle{CjCheadings}

% \begin{table*}[!t]
\vspace {-13mm}


\onecolumn
\zihao{5-}\noindent XXX \hfill 《机器学习》\hfill 2024 年 4 月\\
\noindent\rule[0.25\baselineskip]{\textwidth}{1pt}

{
\centering
\vspace {11mm}
{\zihao{2} \heiti 卷积神经网络与ViT的比较 }

\vskip 5mm

{\zihao{4}\fangsong Koorye}

\vspace {5mm}

\zihao{5}{电子科技大学,计算机科学与工程学院}

}

\vskip 5mm

\zihao{5}{
\setlength{\baselineskip}{16pt}\selectfont{
\noindent {\heiti 摘\quad 要\quad }
在计算机视觉领域,卷积神经网络(CNN)和Vision Transformer模型(ViT)是两种重要的模型。CNN是一种传统的基于卷积操作的神经网络,在图像分类、目标检测等任务中取得了巨大成功。而ViT是一种更为新颖的基于注意力机制的模型,从自然语言处理任务中受到启发,近年来在计算机视觉任务中取得了很好的效果。本文将对这两种模型进行比较,分析它们的优劣势。具体来说,我们将通过比较两者的结构、特征表示、泛化能力等方面,来分析两者的差异。我们将通过基于全局和局部的比较、降维可视化、性能和泛化能力对比等方法,来展示两者的差异。实验结果表明,CNN存在许多意想不到的优势,其特征区域的形状和模式更接近人类大脑的工作模式,同时CNN能更好关注细粒度特征,并具备更强大的泛化能力。希望本文能够为未来的工作带来新的启发。
\par}}
\vspace {5mm}

\zihao{5}{\noindent
{\heiti 关键词 \quad }{卷积神经网络、Vision Transformer。  }
}


\vskip 7mm

%%%%%%%%%%%%%%%%%%%%%%%%%%%%%%%%%%%%%%
\zihao{5}
\vskip 10mm
% \begin{multicols}{1}


%%%%%%%%%%%%%%%%%%%%%%%%%%%%%%%%%%%%%%%%%%
%%%%%%%%%%%%%%%%%%%%%%%%%%%%%%%%%%%%%%%%%%

\include*{chapters/introduction}
\include*{chapters/related_work}
\include*{chapters/preliminaries}
\include*{chapters/method}
\include*{chapters/experiment}
\include*{chapters/conclusion}

\newpage
\vspace {10mm}
\centerline
{\zihao{5}\textsf{参~考~文~献}}
\zihao{5-} \addtolength{\itemsep}{-1em}
\vspace {1.5mm}
\bibliographystyle{gbt7714-numerical}
\bibliography{ref}

\newpage
\include*{chapters/appendix}

\end{document}


