\documentclass{article}
\usepackage[UTF8]{ctex}
\usepackage{algorithm}
\usepackage{algpseudocode}
\usepackage{amsmath}
\usepackage{amssymb}
\usepackage{fancyhdr}
\usepackage{geometry}
\usepackage{sectsty}
\usepackage{tikz}

\xeCJKsetup{CJKmath=true}

\geometry{left=2cm, right=2cm, top=2cm, bottom=2cm}
\sectionfont{\fontsize{12pt}{15pt}\selectfont}

\title{机器学习——第四次作业}
\author{Koorye}
\date{2024年3月22日}

\pagestyle{fancy}
\fancyhead[L]{机器学习——第四次作业}
\fancyhead[R]{Koorye}
\fancyfoot[R]{基于\LaTeX 排版制作}

\begin{document}
\maketitle
\thispagestyle{fancy}

\section{试说明为什么核技巧(kernel trick)使得可以在高维特征空间运用SVM而不显著增加运行时间。}

使用核函数的SVM的决策函数为:

\begin{equation}
    f(x) = \sum_{i=1}^{n} \alpha_i y_i K(x, x_i) + b,
\end{equation}

其中$K(x_i, x_j)=\phi(x_i)^T \phi(x_j)$,$\phi(x)$是$x$的映射函数。

如果直接计算$\phi(x_i)$和$\phi(x_j)$,再计算内积,可能会耗费大量时间。例如,假设$x=(x_1,x_2,\dots,x_n)$,$\phi(x)=(x_1x_1,x_1x_2,\dots,x_2x_1,\dots,x_nx_{n-1},x_nx_n)$。此时需要分别对$x_i,x_j$计算$\phi(x_i),\phi(x_j)$,即

\begin{equation}
    \phi(x_i) = (x_{i1}x_{i1},x_{i1}x_{i2},\dots,x_{in}x_{in}),
    \phi(x_j) = (x_{j1}x_{j1},x_{j1}x_{j2},\dots,x_{jn}x_{jn}),
\end{equation}

再计算内积,即

\begin{equation}
    K(x_i,x_j) = \phi(x_i)^T \phi(x_j) = x_{i1}x_{i1}x_{j1}x_{j1}+\dots+x_{in}x_{in}x_{jn}x_{jn},
\end{equation}

此时需要$3n^2$次计算。

然而,通过核函数可以直接计算内积,即$K(x_i,x_j)=(x_{i1}x_{j1},x_{i2}x_{j2}+\dots+x_{in}x_{jn})^2$。核函数只需要$n$次计算,可以大大减少计算量。

上述核函数是一个二次核函数,属于一种特例。此外,还有很多种核函数,如线性核函数、多项式核函数、高斯核函数等,可以根据具体问题选择合适的核函数。这些核函数的特点是可以直接计算内积,而不需要显式地计算映射函数$\phi(x)$,从而减少计算量。

\section{描述K均值算法,并说明其缺点。}

K均值算法是一种基于距离的聚类算法,其基本思想是:首先初始化K个聚类中心,然后将每个样本分配到与其最近的聚类中心所在的簇中,接着重新计算每个簇的中心,直到聚类中心不再发生变化或者达到最大迭代次数为止。K均值算法可以描述为算法\ref{alg:kmeans}。

\begin{algorithm}[htbp]
    \caption{K均值算法}
    \label{alg:kmeans}
    \begin{algorithmic}[1]
        \Require $N$个样本$x_1,x_2,\dots,x_N, x_i=(x_{i1},x_{i2},\dots,x_{id})$,其中$d$表示维度,聚类数$K$,最大迭代次数$T$
        \Ensure $K$个聚类中心$\mu_1,\mu_2,\dots,\mu_K$
        \State 初始化K个聚类中心$\mu_1,\mu_2,\dots,\mu_K$
        \Repeat
            \For{每个样本$x_i,i\in1,\dots,N$}
                \For{每个聚类中心$\mu_j,j\in1,\dots,K$}
                    \State 计算样本$x_i$与聚类中心$\mu_j$的距离$D(x_i,\mu_j)$
                \EndFor
                \State 为样本$x_i$分配距离最近的聚类中心$C(x_i)=arg\min_jD(x_i,\mu_j)$
            \EndFor
            \For{每个聚类中心$\mu_j,j\in1,\dots,K$}
                \State 重新计算位置$\mu_j=\frac{1}{N_j}\sum_{C(x_i)=j}x_i$,其中$N_j=\sum_{i=1}^{N}I(C(x_i)=j)$
            \EndFor
        \Until{聚类中心不再发生变化或达到最大迭代次数$T$}{}
    \end{algorithmic}
\end{algorithm}

K均值算法的缺点主要有以下几点:

\begin{enumerate}
    \item K均值算法对初始聚类中心敏感,不同的初始聚类中心可能会导致不同的聚类结果。
    \item K均值算法对噪声和异常值敏感,可能会导致聚类结果不稳定。
    \item K均值算法需要事先确定聚类数$K$,但在实际应用中,往往无法事先确定聚类数。
    \item K均值算法只能得到凸形簇,对于非凸形簇的聚类效果不佳。
\end{enumerate}

\section{从EM算法的角度给出KMeans算法和GMM算法的E步和M步,并说明这两种算法的区别和联系。}

EM算法是一种迭代优化算法,用于求解包含隐变量的概率模型的参数估计问题。EM算法包含两个步骤:E步和M步。E步是求期望,即在给定模型参数的情况下,计算隐变量的条件概率分布。M步是求极大值,即在给定隐变量的条件概率分布的情况下,计算模型参数的极大似然估计。

对于KMeans算法来说,E步是将每个样本分配到与其最近的聚类中心所在的簇中,M步是重新计算每个簇的中心。E步可以表示为:

\begin{equation}
    \gamma_{ij} = \begin{cases}
        1, & arg\min_jD(x_i,\mu_j) \\
        0, & \text{Otherwise},
    \end{cases}
\end{equation}

其中$x_i,\mu_j$分别表示第$i$个样本和第$j$个聚类中心。而M步可以表示为:

\begin{equation}
    \mu_j = \frac{\sum_{i=1}^{N}\gamma_{ij}x_i}{\sum_{i=1}^{N}\gamma_{ij}}.
\end{equation}

对于GMM算法来说,E步是计算每个样本属于每个高斯分布的概率,M步是重新计算每个高斯分布的均值和方差。E步可以表示为:

\begin{equation}
    \gamma_{ij} = \frac{\pi_j\mathcal{N}(x_i|\mu_j,\Sigma_j)}{\sum_{k=1}^{K}\pi_k\mathcal{N}(x_i|\mu_k,\Sigma_k)},
\end{equation}

其中$\pi_i,\mu_i,\Sigma_i$分别表示第$i$个高斯分量的系数、均值和方差。而M步可以表示为:

\begin{equation}
    \begin{aligned}
        \mu_j &= \frac{\sum_{i=1}^{N}\gamma_{ij}x_i}{\sum_{i=1}^{N}\gamma_{ij}}, \\
        \Sigma_j &= \frac{\sum_{i=1}^{N}\gamma_{ij}(x_i-\mu_j)(x_i-\mu_j)^T}{\sum_{i=1}^{N}\gamma_{ij}}, \\
        \pi_j &= \frac{1}{N}\sum_{i=1}^{N}\gamma_{ij}.
    \end{aligned}
\end{equation}

KMeans算法和GMM算法的区别主要在于:

\begin{enumerate}
    \item KMeans算法是一种硬聚类算法,每个样本只能属于一个簇;而GMM算法是一种软聚类算法,每个样本可以属于多个高斯分布。
    \item KMeans算法对初始聚类中心敏感,而GMM算法对初始参数不敏感。
    \item KMeans算法对噪声和异常值敏感;而GMM算法对噪声和异常值不敏感。
    \item KMeans算法假设每个簇是一个凸形簇,而GMM算法假设每个高斯分布是一个椭球形簇。
\end{enumerate}

两者的联系主要在于:

\begin{enumerate}
    \item KMeans算法和GMM算法都是基于EM算法的聚类算法,都是通过迭代求最大似然的过程。
    \item KMeans算法和GMM算法求得的都是局部最优解。
    \item KMeans算法和GMM算法都是无监督学习方法,不需要标签。
    \item KMeans算法和GMM算法都需要事先确定聚类数,这难以确定,且会直接影响聚类质量。
\end{enumerate}

\section{写出EM算法的一般形式,试说明其收敛性。}

EM算法是一种迭代优化算法,用于求解包含隐变量的概率模型的参数估计问题。EM算法包含两个步骤:E步和M步。E步是求期望,即在给定模型参数的情况下,计算隐变量的条件概率分布。M步是求极大值,即在给定隐变量的条件概率分布的情况下,计算模型参数的极大似然估计。具体来说,EM算法的一般形式可以表示为算法\ref{alg:em}。

\begin{algorithm}[htbp]
    \caption{EM算法}
    \label{alg:em}
    \begin{algorithmic}[1]
        \Require 观测数据$X=\{x_1,x_2,\dots,x_N\}$,模型参数$\theta$,最大迭代次数$T$
        \Ensure 模型参数$\theta$
        \State 初始化模型参数$\theta_0$
        \Repeat
            \State E步:计算隐变量的条件概率分布$P(Z|X,\theta_t)$
            \State M步:计算极大似然$\theta_{t+1}=arg\max_{\theta}Q(\theta,\theta_t)$,其中$Q(\theta,\theta_t)=\sum_ZP(Z|X,\theta_t)\ln P(X,Z|\theta)$
        \Until{收敛或达到最大迭代次数$T$}{}
    \end{algorithmic}
\end{algorithm}

EM算法之所以能够收敛,是因为EM算法的目标函数是单调递增的。具体来说,通过引入隐变量$Z$的概率分布$q(Z)$,EM算法的目标函数即极大似然可以表示为:

\begin{equation}
        \ln P(X|\theta)=\mathcal{L}(q,\theta)+KL(q||p),
\end{equation}

其中

\begin{equation}
    \mathcal{L}(q,\theta)=\sum_Zq(Z)\ln\frac{P(X,Z|\theta)}{q(Z)}, KL(q||p)=-\sum_Zq(Z)\ln\frac{P(Z|X,\theta)}{q(Z)}.
\end{equation}

由于$KL(q||p)\ge0$,则$\mathcal{L}(q,\theta)$是极大似然的下界。

在E步中,固定$\theta$,最大化下界$\mathcal{L}(q,\theta)$,即通过调整$q$,使得$q$与当前$\theta$下的$P(Z|X,\theta)$尽可能接近。

在M步中,固定$q$,最大化下界$\mathcal{L}(q,\theta)$,即通过调整$\theta$,使得极大似然随下界的增大而增大。

因此EM算法的目标函数是单调递增的,即每次迭代都能使目标函数增大,从而EM算法能够收敛。

\end{document}